\documentclass{article}
\usepackage{geometry}
\usepackage{titling}
\usepackage{hyperref}
\usepackage{amsmath}
\usepackage{amsthm}
\usepackage{amssymb}
\usepackage{graphicx}
\usepackage{caption}
\usepackage{subcaption}
\usepackage{stmaryrd}
\usepackage{enumitem}
\usepackage[dvipsnames]{xcolor}

\geometry{
  a4paper,
  total = {170mm, 257mm},
  left = 20mm,
  top = 20mm,
}
\graphicspath{ {./images/} }

\newtheorem*{axiom}{Axioma}
\renewcommand{\proofname}{Prueba}

\newcommand{\U}{\mathcal{U}}

\newcommand{\none}{\textbf{none}}
\newcommand{\iden}{\textbf{iden}}
\newcommand{\univ}{\textbf{univ}}
\newcommand{\conv}[1]{\ \tilde{}#1}

\title{Trabajo práctico N° 5}
\author{Emanuel Nicolás Herrador}
\date{Mayo 2025}

\makeatletter
\def\@maketitle{%
  \newpage
  \null
  \vskip 1em%
  \begin{center}%
  \let \footnote \thanks
    {\LARGE \@title \par}%
    \vskip 1em%
    {\large \@date}%
  \end{center}%
  \par
  \vskip 1em}
\makeatother

\begin{document}

\maketitle

\noindent\begin{tabular}{@{}ll}
	Estudiante & \theauthor \\
\end{tabular}

\section*{Ejercicio 1}
Se pretende demostrar que $\vDash$ es una relación binaria monótona.
Es decir, se quiere demostrar que $\forall \Gamma, \Gamma'$ conjuntos de fórmulas, $(\Gamma \vDash \varphi) \Rightarrow (\Gamma \cup \Gamma' \vDash \varphi)$.

Sean $\Gamma, \Gamma'$ conjuntos de fórmulas arbitrarios y $\varphi$ una fórmula, supongamos que $\Gamma \vDash \varphi$.
Como sabemos que $\Gamma \vDash \varphi$, entonces $\forall \psi,\ \Gamma \vDash \neg\psi \lor \varphi$, lo que es equivalente a $\Gamma \vDash \psi \Rightarrow \varphi$.
En base a esto, podemos considerar $\forall \psi \in \Gamma'$ que $\Gamma \vDash \psi \Rightarrow \varphi$, por lo que por el lema de deducción $\Gamma \cup \{\psi\} \vDash \varphi$.
Si esto lo hacemos iterativamente para cada fórmula en $\Gamma'$, entonces llegamos a que $\Gamma \cup \Gamma' \vDash \varphi$.
Gracias a esto, entonces, se demuestra. $\qed$

\section*{Ejercicio 2}
Se pretende analizar la afirmación ``si una fórmula es satisfactible, su negación necesariamente no lo es''.
Esto claramente es falso y lo voy a mostrar por contraejemplo.

Sea $\varphi = p$ una proposición atómica, entonces $\varphi$ es satisfactible si $\exists I:\text{PA}\to\{\text{true},\ \text{false}\}$ interpretación tal que $I \vDash \varphi$.
Consideremos $I:\text{PA}\to\{\text{true},\ \text{false}\}$ tal que $I(p) = \text{true}$.
Entonces, claramente $I \vDash \varphi$ por lo que $\varphi$ es satisfactible.

Ahora, consideremos $I':\text{PA}\to\{\text{true},\ \text{false}\}$ tal que $I(p) = \text{false}$.
Entonces, claramente $I \vDash \neg\varphi$ porque $I \not\vDash \varphi$.
Luego, por definición $\neg\varphi$ es también satisfactible.

Con ello, se muestra el contraejemplo y se refuta la afirmación.

\section*{Ejercicio 3}
En este ejercicio se pretende dar la semántica formal de los operadores del álgebra de relaciones.
Para ello, veamos uno por uno.
Sea $\U$ el universo donde se aplican las relaciones:
\begin{itemize}
  \item $\none = \{\} = \emptyset$
  \item $\iden = \{(x, x) : x \in \U\}$
  \item $\univ = \{(x, y) : x, y \in \U \} = \U \times \U$
  \item $\bar{R} = \{(x, y) : (x, y) \notin R\}$
  \item $\conv{R} = \{(x, y) : (y, x) \in R\}$
  \item $R^+ = \{(x, z) : (x, z) \in R \lor (\exists y_1, \dots, y_n \in \U : (x, y_1),\ (y_n, z) \in R \land (\forall 1 \leq i < n : (y_i, y_{i+1}) \in R))\}$
  \item $R^* = \{(x, z) : x = z \lor (x, z) \in R \lor (\exists y_1, \dots, y_n \in \U : (x, y_1),\ (y_n, z) \in R \land (\forall 1 \leq i < n : (y_i, y_{i+1}) \in R))\}$
  \item $R+S = \{(x, y) : (x, y) \in R \lor (x, y) \in S\}$
  \item $R \& S = \{(x, y) : (x, y) \in R \land (x, y) \in S\}$
  \item $R \cdot S = \{(x, z) : \exists y \in \U : (x, y) \in R \land (y, z) \in S\}$
  \item $R-S = \{(x, y) : (x, y) \in R \land (x, y) \notin S\}$
\end{itemize}

\section*{Ejercicio 4}
En este ejercicio, en base a las semánticas definidas en el ejercicio anterior, se pretende demostrar los axiomas dados en la filmina del teórico.
Para ello, veremos cada axioma por separado para mayor claridad.

\begin{axiom}
  $R + \none = R$
\end{axiom}
\begin{proof}
  \begin{equation*}
    \begin{aligned}
      R + \none &= R + \emptyset \\ 
                &= \{(x, y) : (x, y) \in R \lor (x, y) \in \emptyset\} \\ 
                &= \{(x, y) : (x, y) \in R\} \\ 
                &= R
    \end{aligned}
  \end{equation*}
\end{proof}

\begin{axiom} 
  $R \& \univ = R$
\end{axiom}
\begin{proof}
  \begin{equation*}
    \begin{aligned}
      R \& \univ &= R \& (\U \times \U) \\ 
                 &= \{(x, y) : (x, y) \in R \land (x, y) \in \U \times \U\} \\ 
                 &= \{(x, y) : (x, y) \in R\} \\ 
                 &= R
    \end{aligned}
  \end{equation*}
\end{proof}

\begin{axiom}
  $R+S = S+R$
\end{axiom}
\begin{proof}
  \begin{equation*}
    \begin{aligned}
      R+S &= \{(x, y) : (x, y) \in R \lor (x, y) \in S\} \\ 
          &= \{(x, y) : (x, y) \in S \lor (x, y) \in R\} \\ 
          &= S+R
    \end{aligned}
  \end{equation*}
\end{proof}

\begin{axiom}
  $R \& S = S \& R$
\end{axiom}
\begin{proof}
  \begin{equation*}
    \begin{aligned}
      R \& S &= \{(x, y) : (x, y) \in R \land (x, y) \in S\} \\ 
             &= \{(x, y) : (x, y) \in S \land (x, y) \in R\} \\ 
             &= S \& R 
    \end{aligned}
  \end{equation*}
\end{proof}

\begin{axiom}
  $R + R = R$
\end{axiom}
\begin{proof}
  \begin{equation*}
    \begin{aligned}
      R + R &= \{(x, y) : (x, y) \in R \lor (x, y) \in R\} \\ 
            &= \{(x, y) : (x, y) \in R\} \\ 
            &= R
    \end{aligned}
  \end{equation*}
\end{proof}

\begin{axiom}
  $R \& R = R$
\end{axiom}
\begin{proof}
  \begin{equation*}
    \begin{aligned}
      R \& R &= \{(x, y) : (x, y) \in R \land (x, y) \in R\} \\ 
             &= \{(x, y) : (x, y) \in R\} \\ 
             &= R
    \end{aligned}
  \end{equation*}
\end{proof}

\begin{axiom}
  $\overline{\univ} = \none$
\end{axiom}
\begin{proof}
  \begin{equation*}
    \begin{aligned}
      \overline{\univ} &= \{(x, y) : (x, y) \notin \U \times \U\} \\ 
                       &= \emptyset \\ 
                       &= \none
    \end{aligned}
  \end{equation*}
\end{proof}

\begin{axiom}
  $R + \bar{R} = \univ$
\end{axiom}
\begin{proof}
  \begin{equation*}
    \begin{aligned}
      R + \bar{R} &= R + \{(x, y) : (x, y) \notin R\} \\ 
                  &= \{(x, y) : (x, y) \in R \lor (x, y) \notin R\} \\ 
                  &= \{(x, y) \in \U \times \U\} \\ 
                  &= \U \times \U \\ 
                  &= \univ
    \end{aligned}
  \end{equation*}
\end{proof}

\begin{axiom}
  $R \& \bar{R} = \none$
\end{axiom}
\begin{proof}
  \begin{equation*}
    \begin{aligned}
      R \& \bar{R} &= R \& \{(x, y) : (x, y) \notin R\} \\ 
                   &= \{(x, y) : (x, y) \in R \land (x, y) \notin R\} \\ 
                   &= \emptyset \\ 
                   &= \none
    \end{aligned}
  \end{equation*}
\end{proof}

\begin{axiom}
  $\bar{\bar{R}} = R$
\end{axiom}
\begin{proof}
  \begin{equation*}
    \begin{aligned}
      \bar{\bar{R}} &= \overline{\{(x, y) : (x, y) \notin R\}} \\ 
                    &= \{(x, y) : (x, y) \in R\} \\ 
                    &= R
    \end{aligned}
  \end{equation*}
\end{proof}

\begin{axiom}
  $(R+S)\& T = (R \& T) + (S \& T)$
\end{axiom}
\begin{proof}
  \begin{equation*}
    \begin{aligned}
      (R+S) \& T &= \{(x, y) : (x, y) \in R \lor (x, y) \in S\} \& T \\ 
                 &= \{(x, y) : ((x, y) \in R \lor (x, y) \in S) \land (x, y) \in T\} \\ 
                 &= \{(x, y) : ((x, y) \in R \land (x, y) \in T) \lor ((x, y) \in S \land (x, y) \in T)\} \\ 
                 &= \{(x, y) : (x, y) \in R \land (x, y) \in T\} + \{(x, y) : (x, y) \in S \land (x, y) \in T\} \\ 
                 &= (R \& T) + (S \& T)
    \end{aligned}
  \end{equation*}
\end{proof}

\begin{axiom}
  $(R \& S) + T = (R+T) \& (S+T)$
\end{axiom}
\begin{proof}
  \begin{equation*}
    \begin{aligned}
      (R \& S) + T &= \{(x, y) : (x, y) \in R \land (x, y) \in S\} + T \\ 
                   &= \{(x, y) : ((x, y) \in R \land (x, y) \in S) \lor (x, y) \in T\} \\ 
                   &= \{(x, y) : ((x, y) \in R \lor (x, y) \in T) \land ((x, y) \in S \lor (x, y) \in T)\} \\ 
                   &= \{(x, y) : (x, y) \in R \lor (x, y) \in T\} \& \{(x, y) : (x, y) \in S \lor (x, y) \in T\} \\ 
                   &= (R+S) \& (S+T)
    \end{aligned}
  \end{equation*}
\end{proof}

\begin{axiom}
  $R \cdot \iden = \iden \cdot R = R$
\end{axiom}
\begin{proof}
  \begin{equation*}
    \begin{aligned}
      R \cdot \iden &= R \cdot \{(x, x) : x \in \U\} \\ 
                    &= \{(x, z) : \exists y \in \U : (x, y) \in R \land (y, z) \in \{(k, k) : k \in \U\}\} \\ 
                    &= \{(x, z) : \exists y \in \U : (x, y) \in R \land y = z\} \\ 
                    &= \{(x, z) : (x, z) \in R\} \\ 
                    &= R \\ 
                    \\ 
      \iden \cdot &= \{(x, x) : x \in \U\} \cdot R \\ 
                    &= \{(x, z) : \exists y \in \U : (x, y) \in \{(k, k) : k \in \U\} \land (z, y) \in R\} \\ 
                    &= \{(x, z) : \exists y \in \U : x = y \land (z, y) \in R\} \\ 
                    &= \{(x, z) : (x, y) \in R\} \\ 
                    &= R
    \end{aligned}
  \end{equation*}
\end{proof}

\begin{axiom}
  $(R \cdot S) \cdot T = R \cdot (S \cdot T)$
\end{axiom}
\begin{proof}
  \begin{equation*}
    \begin{aligned}
      (R \cdot S) \cdot T &= \{(x, y) : \exists k_0 \in \U : (x, k_0) \in R \land (k_0, y) \in S\} \cdot T \\ 
                          &= \{(x, y) : \exists k_1 \in \U : (\exists k_0 \in \U : (x, k_0) \in R \land (k_0, k_1) \in S) \land (k_1, y) \in T\} \\ 
                          &= \{(x, y) : \exists k_0, k_1 \in \U : (x, k_0) \in R \land (k_0, k_1) \in S \land (k_1, y) \in T\} \\ 
                          &= \{(x, y) : \exists k_0 \in \U : (x, k_0) \in R \land (\exists k_1 \in \U: (k_0, k_1) \in S \land (k_1, y) \in T)\} \\ 
                          &= R \cdot \{(x, y) : \exists k_1 \in \U : (x, k_1) \in S \land (k_1, y) \in T\} \\ 
                          &= R \cdot (S \cdot T)
    \end{aligned}
  \end{equation*}
\end{proof}

\begin{axiom}
  $(R+S) \cdot T = (R \cdot T) + (S \cdot T)$
\end{axiom}
\begin{proof}
  \begin{equation*}
    \begin{aligned}
      (R+S) \cdot T &= \{(x, y) : (x, y) \in R \lor (x, y) \in S\} \cdot T \\ 
                    &= \{(x, y) : \exists k \in \U : ((x, k) \in R \lor (x, k) \in S) \land (k, y) \in T\} \\ 
                    &= \{(x, y) : \exists k \in \U : ((x, k) \in R \land (k, y) \in T) \lor ((x, k) \in S \land (k, y) \in T)\} \\ 
                    &= \{(x, y) : \exists k \in \U : (x, k) \in R \land (k, y) \in T\} + \{(x, y) : \exists k \in \U : (x, k) \in S \land (k, y) \in T\} \\ 
                    &= (R \cdot T) + (S \cdot T)
    \end{aligned}
  \end{equation*}
\end{proof}

\begin{axiom}
  $\conv{(\conv{R})} = R$
\end{axiom}
\begin{proof}
  \begin{equation*}
    \begin{aligned}
      \conv{(\conv{R})} &= \{(x, y) : (y, x) \in \conv{R}\} \\ 
                        &= \{(x, y) : (y, x) \in \{(a, b) : (b, a) \in R\}\} \\ 
                        &= \{(x, y) : (x, y) \in R\} \\ 
                        &= R
    \end{aligned}
  \end{equation*}
\end{proof}

\begin{axiom}
  $\conv{(R \cdot S)} = \conv{S} \cdot \conv{R}$
\end{axiom}
\begin{proof}
  \begin{equation*}
    \begin{aligned}
      \conv{(R \cdot S)} &= \{(x, y) : (y, x) \in R \cdot S\} \\ 
                         &= \{(x, y) : (y, x) \in \{(a, b) : \exists c \in \U : (a, c) \in R \land (c, b) \in S\}\} \\ 
                         &= \{(x, y) : \exists k \in \U : (y, k) \in R \land (k, x) \in S\} \\ 
                         &= \{(x, y) : \exists k \in \U : (x, k) \in \conv{S} \land (k, y) \in \conv{R}\} \\ 
                         &= \conv{S} \cdot \conv{R}
    \end{aligned}
  \end{equation*}
\end{proof}

\begin{axiom}
  $\conv{(R+S)} = \conv{R}+\conv{S}$
\end{axiom}
\begin{proof}
  \begin{equation*}
    \begin{aligned}
      \conv{(R + S)} &= \{(x, y) : (y, x) \in R + S\} \\ 
                     &= \{(x, y) : (y, x) \in R \lor (y, x) \in S\} \\ 
                     &= \{(x, y) : (x, y) \in \conv{R} \lor (x, y) \in \conv{S}\} \\ 
                     &= \conv{R} + \conv{S}
    \end{aligned}
  \end{equation*}
\end{proof}

\begin{axiom}
  $\overline{\conv{R}} = \conv{(\bar{R})}$
\end{axiom}
\begin{proof}
  \begin{equation*}
    \begin{aligned}
      \overline{\conv{R}} &= \{(x, y) : (x, y) \notin \conv{R}\} \\ 
                          &= \{(x, y) : (x, y) \notin \{(a, b) : (b, a) \in R\}\} \\ 
                          &= \{(x, y) : (y, x) \notin R\} \\ 
                          &= \{(x, y) : (y, x) \in \overline{R}\} \\ 
                          &= \conv{(\overline{R})}
    \end{aligned}
  \end{equation*}
\end{proof}

\begin{axiom}
  $R^+ = R \cdot R^+ + R$
\end{axiom}
\begin{proof}
  \begin{equation*}
    \begin{aligned}
      R^+ &= \{(x, y) : (x, y) \in R \lor (\exists k_1, \dots k_n \in \U : (x, k_1), (k_n, y) \in R \land (\forall 1 \leq i < n : (k_i, k_{i+1}) \in R))\} \\ 
          &= R + \{(x, y) : \exists k_1, \dots k_n \in \U : (x, k_1), (k_n, y) \in R \land (\forall 1 \leq i < n : (k_i, k_{i+1}) \in R)\} \\ 
          &= \{(x, y) : \exists k_1 \in \U : (x, k_1) \in R \land \\
          & \qquad\qquad ((k_1, y) \in R \lor \exists k_2, \dots, k_n \in \U : (k_1, k_2), (k_n, y) \in R \land (\forall 2 \leq i < n : (k_i, k_{i+1}) \in R))\} + R \\
          &= R \cdot \{(x, y) : (x, y) \in R \lor \exists k_1, \dots, k_n \in \U : (x, k_1), (k_n, y) \in R \land (\forall 1 \leq i < n : (k_i, k_{i+1}) \in R)\} + R \\ 
          &= R \cdot R^+ + R
    \end{aligned}
  \end{equation*}
\end{proof}

\begin{axiom}
  $R^* = (R \cdot R^*) + \iden$
\end{axiom}
\begin{proof}
  \begin{equation*}
    \begin{aligned}
      R^* &= \{(x, y) : x = y \lor (x, y) \in R \lor (\exists k_1, \dots, k_n \in \U : (x, k_1), (k_n, y) \in R \land (\forall 1 \leq i < n : (k_i, k_{i+1}) \in R))\} \\ 
          &= \{(x, y) : x = y\} + \\ 
          & \qquad\qquad \{(x, y) : (x, y) \in R \lor (\exists k_1, \dots, k_n \in \U : (x, k_1), (k_n, y) \in R \land (\forall 1 \leq i < n : (k_i, k_{i+1}) \in R))\} \\ 
          &= \iden + \{(x, y) : \exists k_1 \in \U : (x, k_1) \in R \land \\ 
          & \qquad\qquad (k_1 = y \lor (k_1, y) \in R \lor \exists k_2, \dots, k_n \in \U : (k_1, k_2), (k_n, y) \in R \land (\forall 2 \leq i < n : (k_i, k_{i+1}) \in R))\} \\ 
          &= R \cdot \{(x, y) : x = y \lor (x, y) \in R \lor \\ 
          & \qquad\qquad (\exists k_1, \dots, k_n \in \U : (x, k_1), (k_n, y) \in R \land (\forall 1 \leq i < n : (k_i, k_{i+1}) \in R))\} + \iden \\ 
          &= (R \cdot R^*) + \iden
    \end{aligned}
  \end{equation*}
\end{proof}

\section*{Ejercicio 5}
Ahora, queremos caracterizar cuándo una relación $R$ es un orden parcial, un orden total y un orden estricto.
Veamos cada una en un item separado.

\subsection*{Item A}
Respecto al orden parcial, sabemos que $R$ es un orden parcial si es reflexiva, antisimétrica y transitiva.
Es decir, si cumple las siguientes condiciones:
\begin{itemize}
  \item $\forall x \in \U : (x, x) \in R$
  \item $\forall x, y \in \U : (x, y) \in R \land (y, x) \in R \Rightarrow x = y$
  \item $\forall x, y, z \in \U : (x, y) \in R \land (y, z) \in R \Rightarrow (x, z) \in R$
\end{itemize}

Veamos cada una de ellas por separado comenzando con la reflexividad:
\begin{equation*}
  \begin{aligned}
    \forall x \in \U : (x, x) \in R &\equiv \forall (x, x) \in \iden : (x, x) \in R \\ 
                                    &\equiv \forall k \in \iden : k \in R \\ 
                                    &\equiv \iden \subseteq R
  \end{aligned}
\end{equation*}

Respecto a la antisimetría, tenemos:
\begin{equation*}
  \begin{aligned}
    \forall x, y \in \U : (x, y) \in R \land (y, x) \in R \Rightarrow x = y &\equiv \forall x, y \in \U : (x, y) \in R \land (x, y) \in \conv{R} \Rightarrow (x, y) \in \iden \\ 
                                                                            &\equiv \forall x, y \in \U : (x, y) \in R \& \conv{R} \Rightarrow (x, y) \in \iden \\ 
                                                                            &\equiv R \& \conv{R} \subseteq \iden
  \end{aligned}
\end{equation*}

Y, para transitividad:
\begin{equation*}
  \begin{aligned}
    \forall x, y, z \in \U : (x, y) \in R \land (y, z) \in R \Rightarrow (x, z) \in R &\equiv \forall x, z \in \U : (x, z) \in R \cdot R \Rightarrow (x, z) \in R \\ 
                                                                                      &\equiv R \cdot R \subseteq R
  \end{aligned}
\end{equation*}

Por ello, dado $R_R = \iden \subseteq R$, $R_A = R \& \conv{R} \subseteq \iden$ y $R_T = R \cdot R \subseteq R$, entonces el conjunto de ecuaciones que especifican que $R$ es un orden parcial es $R_\text{poset} = \{R_R, R_A, R_T\}$.

\subsection*{Item B}
Ahora, queremos ver el conjunto de ecuaciones que establecen cuándo $R$ es un orden total.
Recordemos que un orden total es un orden parcial tal que $\forall x, y \in \U : (x, y) \in R \lor (y, x) \in R$.
Para ello, veamos que:
\begin{equation*}
  \begin{aligned}
    \forall x, y \in \U : (x, y) \in R \lor (y, x) \in R &\equiv \forall (x, y) \in \U \times \U : (x, y) \in R \lor (x, y) \in \conv{R} \\ 
                                                         &\equiv \forall (x, y) \in \U \times \U : (x, y) \in R + \conv{R} \\ 
                                                         &\equiv \U \times \U = R + \conv{R} \\ 
                                                         &\equiv \univ = R + \conv{R}
  \end{aligned}
\end{equation*}

Por esto, sea $R_C = (\univ = R + \conv{R})$, entonces el conjunto que establece que $R$ es un orden total es $R_\text{ord. total} = R_\text{poset} \cup R_C$

\subsection*{Item C}
Como último punto, ahora, queremos ver cuál sería la caracterización de las condiciones para un orden estricto en el álgebra relacional.
Recordemos que $R$ es un orden estricto si es asimétrica y transitiva.
$R$ es asimétrica si $\forall a, b \in \U : (a, b) \in R \Rightarrow (b, a) \notin R$.
Veamos entonces que esto equivale a que:
\begin{equation*}
  \begin{aligned}
    \forall a, b \in \U : (a, b) \in R \Rightarrow (b, a) \notin R &\equiv \forall a, b \in \U : (a, b) \in R \Rightarrow (a, b) \in \overline{\conv{R}} \\ 
                                                                   &\equiv R \subseteq \overline{\conv{R}}
  \end{aligned}
\end{equation*}

Por ello, tenemos que sea $R_{A2} = R \subseteq \overline{\conv{R}}$, entonces $R_\text{ord. estricto} = \{R_{A2}, R_T\}$.

\section*{Ejercicio 6}
Ahora queremos dar conjuntos de ecuaciones respecto a una relación en un sistema de transiciones etiquetadas.
Para ello, consideraremos $Act$ al conjunto de acciones o eventos y $T_a$ la relación de transición etiquetada con $a \in Act$ en un sistema de transiciones etiquetadas.
Veamos cada punto por separado.

\subsection*{Item A}
Se pretende caracterizar la simulación.
Recordemos que $R$ es una simulación si cumple que:
\begin{equation*}
  \begin{aligned}
    &\qquad \forall (s, t) \in R : \forall s' \in \U,\ a \in Act : ((s, s') \in T_a \Rightarrow \exists t' \in \U : (t, t') \in T_a \land (s', t') \in R) \equiv \\ 
    &\equiv \forall s, t, s' \in \U,\ a \in Act : ((t, s) \in \conv{R} \land (s, s') \in T_a \Rightarrow \exists t' \in \U : (t, t') \in T_a \land (t', s') \in \conv{R}) \\ 
    &\equiv \forall s, t, s' \in \U,\ a \in Act : (t, s') \in \conv{R} \cdot T_a \Rightarrow (t, s') \in T_a \cdot \conv{R} \\ 
    &\equiv \forall a \in Act : \conv{R} \cdot T_a \subseteq T_a \cdot \conv{R}
  \end{aligned}
\end{equation*}

Luego, entonces, sea $S_a = \conv{R} \cdot T_a \subseteq T_a \cdot \conv{R}$ para $a \in Act$, tenemos que el conjunto de ecuaciones que especifica cuándo $R$ es una simulación está dado por $S = \{S_a : a \in Act\}$.

\subsection*{Item B}
Ahora, se pretende caracterizar la bisimulación.
Para ello, recordemos que $R$ es una bisimulación si tanto $R$ como $R^{-1}$ son simulaciones.
Como $R^{-1}$ es la conversa $\conv{R}$, entonces podemos ver que la condición de bisimulación para una transición $a$ en específico queda igual a:
\begin{equation*}
  \begin{aligned}
    \conv{R} \cdot T_a &\subseteq T_a \cdot \conv{R} &\text{porque }R\text{ es sim.} \\ 
    \\ 
    \conv{(R^{-1})} \cdot T_a &\subseteq T_a \cdot \conv{(R^{-1})} &\text{porque } R^{-1}\text{ es sim.} \\ 
    \conv{(\conv{R})} \cdot T_a &\subseteq T_a \cdot \conv{(\conv{R})} \\ 
    R \cdot T_a &\subseteq T_a \cdot R 
  \end{aligned}
\end{equation*}

Por esto, sea $S^{-1}_a = R \cdot T_a \subseteq T_a \cdot R$ para $a \in Act$, tenemos que el conjunto de ecuaciones que especifica que $R$ es bisimulación está dado por $BF = \{S_a : a \in Act\} \cup \{S^{-1}_a : a \in Act\}$.

\subsection*{Item C}
Ahora, se pretende caracterizar la bisimulación débil.
Para esto, recordemos primero que una relación $R$ es una bisimulación débil si es una bisimulación en donde las transiciones $\tau$ son ``invisibles''.
Primero, para mayor facilidad, veamos cuándo $R$ es una simulación débil.
Tenemos que, en este caso:
\begin{equation*}
  \begin{aligned}
    &\qquad \forall (s, t) \in R : \forall s' \in \U,\ a \in Act \neq \tau : ((s, s') \in T_a \Rightarrow \exists t' \in \U : (t, t') \in T_\tau^* \cdot T_a \cdot T_\tau^* \land (s', t') \in R) \\ 
    &\equiv \forall s, t \in \U,\ a \neq \tau \in Act : ((t, s') \in \conv{R} \cdot T_a \Rightarrow (t, s') \in T_\tau^* \cdot T_a \cdot T_\tau^* \cdot \conv{R}) \\
    &\equiv \forall a \neq \tau \in Act : \conv{R} \cdot T_a \subseteq T_\tau^* \cdot T_a \cdot T_\tau^* \cdot \conv{R}
  \end{aligned}
\end{equation*}

Notar que al usar la clausura reflexo-transitiva, entonces se considera hacer $0, 1$ o más acciones silenciosas antes y después de hacer la de $a$.
Esta condición para este tipo de simulación débil la llamaremos $SD_a = \{\conv{R} \cdot T_a \subseteq T_\tau^* \cdot T_a \cdot T_\tau^* \cdot \conv{R} : a \in Act \smallsetminus \{\tau\}\}$.

Ahora, si consideramos la transición $\tau$, se puede simular haciendo $0, 1$ o más acciones silenciosas en total.
Por ello:
\begin{equation*}
  \begin{aligned}
    &\qquad \forall (s, t) \in R : \forall s' \in \U : ((s, s') \in T_\tau \Rightarrow \exists t' \in \U : (t, t') \in T_\tau^* \land (s', t') \in R) \\ 
    &\equiv \conv{R} \cdot T_\tau \subseteq T_\tau^* \cdot \conv{R}
  \end{aligned}
\end{equation*}

Luego, consideramos el otro caso como $SD_\tau = \{\conv{R} \cdot T_\tau \subseteq T_\tau^* \cdot \conv{R}\}$.

Finalmente, como la bisimulación débil requiere que $R$ y $R^{-1}$ sean simulaciones débiles, del mismo modo que en el ejercicio anterior llegamos a:
\begin{equation*}
  \begin{aligned}
    SD_a^{-1} &= \{R \cdot T_a \subseteq T_\tau^* \cdot T_a \cdot T_\tau^* \cdot R : a \in Act \smallsetminus \{\tau\}\} \\ 
    SD_\tau^{-1} &= \{R \cdot T_\tau \subseteq T_\tau^* \cdot R\}
  \end{aligned}
\end{equation*}

Luego, el conjunto de ecuaciones para establecer que $R$ es una bisimulación débil es:
\begin{equation*}
  BD = SD_a \cup SD_\tau \cup SD^{-1}_a \cup SD^{-1}_\tau
\end{equation*}

\section*{Ejercicio 7}
Como último ejercicio, se pretende caracterizar el conjunto de ecuaciones necesarias para dar propiedades acerca de un grafo.
Para ello, consideraremos un grafo dirigido con conjunto de nodos $N$ y $E \subseteq N \times N$ su relación de aristas.
Con esto en mente, veamos cada propiedad por separado.

\subsection*{Item A}
Se pretende caracterizar que el grafo es acíclico.
Para ello, veamos que:
\begin{equation*}
  \begin{aligned}
    &\qquad \forall x \in N : (x, x) \notin E \land \nexists k_1, \dots, k_n \in N : (x, k_1), (k_n, x) \in E \land (\forall 1 \leq i < n : (k_i, k_{i+1}) \in E) \\ 
    &\equiv \forall x \in N : \neg((x, x) \in E \lor \exists k_1, \dots, k_n \in N : (x, k_1), (k_n, x) \in E \land (\forall 1 \leq i < n : (k_i, k_{i+1}) \in E)) \\ 
    &\equiv \forall x \in N : \neg((x, x) \in E^+) \\ 
    &\equiv \forall x \in N : (x, x) \in \overline{E^+} \\ 
    &\equiv \forall k \in \iden : k \in \overline{E^+} \\ 
    &\equiv \iden \subseteq \overline{E^+}
  \end{aligned}
\end{equation*}

Luego, entonces, para ver si el grafo es acíclico tenemos $G_\text{acíclico} = \{\iden \subseteq \overline{E^+}\}$.

\subsection*{Item B}
Se pretende caracterizar que el grafo es no dirigido.
Para ello, veamos que:
\begin{equation*}
  \begin{aligned}
    &\qquad \forall x, y \in N : (x, y) \in E \iff (y, x) \in E \\ 
    &\equiv \forall (x, y) \in N \times N : (x, y) \in E \iff (x, y) \in \conv{E} \\ 
    &\equiv E = \conv{E}
  \end{aligned}
\end{equation*}

Luego, $G_\text{no dirigido} = \{E = \conv{E}\}$.

\subsection*{Item C}
Se pretende caracterizar que el grafo es fuertemente conexo.
Para ello, veamos que:
\begin{equation*}
  \begin{aligned}
    &\qquad \forall x, y \in N : x = y \lor (x, y) \in E \lor (\exists k_1, \dots, k_n \in N : (x, k_1), (k_n, y) \in E \land (\forall 1 \leq i < n : (k_i, k_{i+1}) \in E)) \\ 
    &\equiv \forall x, y \in N : (x, y) \in E^* \\ 
    &\equiv \forall (x, y) \in N \times N : (x, y) \in E^* \\ 
    &\equiv \univ \subseteq E^* \\ 
    &\equiv \univ = E^*
  \end{aligned}
\end{equation*}

Luego, $G_\text{fuertemente conexo} = \{\univ = E^*\}$

\subsection*{Item D}
Se pretende caracterizar que el grafo es conexo.
Para ello, veamos que:
\begin{equation*}
  \begin{aligned}
    &\qquad \forall x, y \in N : x = y \lor \\ 
    & \qquad\qquad ((x, y) \in E \lor (\exists k_1, \dots, k_n \in N : (x, k_1), (k_n, y) \in E \land (\forall 1 \leq i < n : (k_i, k_{i+1}) \in E))) \lor \\ 
    & \qquad\qquad ((y, x) \in E \lor (\exists k_1, \dots, k_n \in N : (y, k_1), (k_n, x) \in E \land (\forall 1 \leq i < n : (k_i, k_{i+1}) \in E))) \\ 
    &\equiv \forall (x, y) \in N \times N : \\ 
    &\qquad\qquad (x = y \lor (x, y) \in E \lor (\exists k_1, \dots, k_n \in N : (x, k_1), (k_n, y) \in E \land (\forall 1 \leq i < n : (k_i, k_{i+1}) \in E))) \lor \\ 
    &\qquad\qquad (y = x \lor (y, x) \in E \lor (\exists k_1, \dots, k_n \in N : (y, k_1), (k_n, x) \in E \land (\forall 1 \leq i < n : (k_i, k_{i+1}) \in E))) \\
    &\equiv \forall (x, y) \in \univ : (x, y) \in E^* \lor (y, x) \in E^* \\ 
    &\equiv \forall (x, y) \in \univ : (x, y) \in E^* \lor (x, y) \in \conv{(E^*)} \\ 
    &\equiv \forall (x, y) \in \univ : (x, y) \in E^* + \conv{(E^*)} \\ 
    &\equiv \univ \subseteq E^* + \conv{(E^*)} \\ 
    &\equiv \univ = E^* + \conv{(E^*)}
  \end{aligned}
\end{equation*}

Luego, $G_\text{conexo} = \{\univ = E^* + \conv{(E^*)}\}$

\subsection*{Item E}
Finalmente, y como último item, queremos caracterizar al grafo como un árbol.
En este caso, dado que consideramos un grafo dirigido, supongo que se pretende caracterizar cuándo este es un árbol dirigido.
Por esto, no va a ser suficiente garantizar que sea acíclico y conexo porque eso lo cumple todo DAG.

Para asegurar que un DAG sea un árbol, se debe garantizar que este tenga una raíz, i.e. un solo vértice sin aristas de entrada.
Es decir:
\begin{equation*}
  \begin{aligned}
    &\exists! x \in N : \forall y \in N : (y, x) \notin E \\ 
    &(\exists x \in N : \forall y \ in N : (y, x) \notin E) \land (\forall x \in N : \forall y \in N : (\forall z \in N : (z, x) \notin E) \land (\forall z \in N : (z, y) \notin E) \Rightarrow x = y)
  \end{aligned}
\end{equation*}

Luego, veamos cada predicado por separado para mayor facilidad:
\begin{equation*}
  \begin{aligned}
    &\exists x \in N : \forall y \in N : (y, x) \notin E \\ 
    &\neg\neg(\exists x \in N : \forall y \in N : (y, x) \notin E) \\ 
    &\neg(\forall x \in N : \exists y \in N : (y, x) \in E) \\ 
    &\neg(\forall x \in N : \exists y \in N : (x, y) \in \conv{E} \land (y, x) \in E) \\ 
    &\neg(\forall x \in N : (x, x) \in \conv{E} \cdot E) \\ 
    &\neg(\iden \subseteq \conv{E} \cdot E) \\ 
    &\iden \not\subseteq \conv{E} \cdot E
  \end{aligned}
\end{equation*}

Y, ahora respecto al segundo predicado:
\begin{equation*}
  \begin{aligned}
    &\forall x \in N : \forall y \in N : (\forall z \in N : (z, x) \notin E) \land (\forall z \in N : (z, y) \notin E) \Rightarrow x = y \\ 
    &\forall x \in N : \forall y \in N : \neg((\forall z \in N : (z, x) \notin E) \land (\forall z \in N : (z, y) \notin E)) \lor x = y \\ 
    &\forall x \in N : \forall y \in N : (\exists z \in N : (z, x) \in E) \lor (\exists z \in N : (z, y) \in E) \lor x = y \\ 
    &\forall x \in N : \forall y \in N : (\exists z \in N : (x, z) \in \conv{E} \land (z, x) \in E) \lor (\exists z \in N : (y, z) \in \conv{E} \land (z, y) \in E) \lor (x, y) \in \iden \\ 
    &\forall x \in N : \forall y \in N : (x, x) \in \conv{E} \cdot E \lor (y, y) \in \conv{E} \cdot E \lor (x, y) \in \iden \\ 
    &\neg\neg(\forall x \in N : \forall y \in N : (x, x) \in \conv{E} \cdot E \lor (y, y) \in \conv{E} \cdot E \lor (x, y) \in \iden) \\ 
    &\neg(\exists x \in N : \exists y \in N : (x, x) \notin \conv{E} \cdot E \land (y, y) \notin \conv{E} \cdot E \land (x, y) \notin \iden) \\ 
    &\neg(\exists x \in N : \exists y \in N : (x, x) \in \overline{\conv{E} \cdot E} \land (y, y) \in \overline{\conv{E} \cdot E} \land (x, y) \in \overline{\iden}) \\ 
  \end{aligned}
\end{equation*}

\begin{quotation}
  \textcolor{red}{\textbf{Nota:}} ¿Cómo debería continuar para hacer la reducción de forma correcta?
\end{quotation}

\end{document}
